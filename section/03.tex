\section{第三方模块}
\texttt{hi-nginx}目前集成了一些常用的第三方模块。包括:
\begin{itemize}
\item array-var-nginx-module
\item echo-nginx-module
\item form-input-nginx-module
\item headers-more-nginx-module
\item iconv-nginx-module
\item memc-nginx-module
\item nchan
\item nginx-http-concat
\item nginx-push-stream-module
\item nginx-rtmp-module
\item nginx-upload-module
\item ngx_coolkit
\item ngx_devel_kit
\item rds-csv-nginx-module
\item rds-json-nginx-module
\item redis2-nginx-module
\item set-misc-nginx-module
\item srcache-nginx-module
\item xss-nginx-module
\end{itemize}
它们都放在\path{3rd}目录中,需要时,只需\texttt{--add-module=3rd/xxx}即可。

如果用户有需要集成的其他模块,最好也放在\path{3rd}目录中,便于统一管理。然后,依样画葫芦即可。

\section{模块开发}
一般而言,\texttt{nginx}模块开发以\texttt{c}语言为基础。但是,对\texttt{hi-nginx}而言,\texttt{c}和\texttt{c++}都是完全支持的。\texttt{hi-nginx}本身即以\texttt{c++}模块\texttt{ngx_http_hi_module}为基础,它可以作为用户使用\texttt{c++}为\texttt{hi-nginx}开发自定义模块的范例演示。

实际上,把\texttt{ngx_http_hi_module}放在\path{3rd}目录中也是完全可以的。

\section{演示代码}
\texttt{hi-nginx}配有较为完整的演示代码。请参考\url{https://github.com/webcpp/hi_demo}



